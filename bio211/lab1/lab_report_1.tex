%%%%%%%%%%%%%%%%%%%%%%%%%%%%%%%%%%%%%%%%%
% University/School Laboratory Report
% LaTeX Template
% Version 3.0 (4/2/13)
%
% This template has been downloaded from:
% http://www.LaTeXTemplates.com
%
% Original author:
% Linux and Unix Users Group at Virginia Tech Wiki 
% (https://vtluug.org/wiki/Example_LaTeX_chem_lab_report)
%
% License:
% CC BY-NC-SA 3.0 (http://creativecommons.org/licenses/by-nc-sa/3.0/)
%
%%%%%%%%%%%%%%%%%%%%%%%%%%%%%%%%%%%%%%%%%

%----------------------------------------------------------------------------------------
%	PACKAGES AND DOCUMENT CONFIGURATIONS
%----------------------------------------------------------------------------------------

\documentclass{article}

\usepackage{mhchem} % Package for chemical equation typesetting
\usepackage{siunitx} % Provides the \SI{}{} command for typesetting SI units
\usepackage{sympytex}
\usepackage{python}
\usepackage{astron}
\usepackage[left=3cm, top=3cm, bottom=3cm, right=3cm]{geometry}


\usepackage{graphicx} % Required for the inclusion of images

\setlength\parindent{0pt} % Removes all indentation from paragraphs

\renewcommand{\labelenumi}{\alph{enumi}.} % Make numbering in the enumerate environment by letter rather than number (e.g. section 6)

\usepackage{times} % Uncomment to use the Times New Roman font

%----------------------------------------------------------------------------------------
%	DOCUMENT INFORMATION
%----------------------------------------------------------------------------------------

\title{Fitness Lab} % Title

\author{Jacques \textsc{Uber}} % Author name

\date{\today} % Date for the report

\begin{document}

\maketitle % Insert the title, author and date

\begin{center}
\begin{tabular}{l r}
Date Performed: & October 30, 2013 \\ % Date the experiment was performed
Partners: & Jacques Uber \\ % Partner names
& n/a \\
Instructor: & Arick Rouhe % Instructor/supervisor
\end{tabular}
\end{center}

% If you wish to include an abstract, uncomment the lines below
% \begin{abstract}
% Abstract text
% \end{abstract}

%----------------------------------------------------------------------------------------
%	SECTION 1
%----------------------------------------------------------------------------------------
\section{Introduction}
% hypothesis - specific prediction
% Caffeine drinkers take a longer time to return to their resting heart rate than non-caffeine drinkers.

We predict that if caffeine drinkers and non-caffeine drinkers are both exposed to a cardiovascular
fitness test, then caffeine drinkers will take longer to recover to their resting heart rate than
non-caffeine drinkers. \\

%----------------------------------------------------------------------------------------
%	SECTION 2
%----------------------------------------------------------------------------------------
\section{Methods}
\label{sec:methods}
% Description of what we did in paragraph form. Detail is important.
% Things to talk about:
%   * How did we time?
%   * How did we use the stopwatch?
%   * How hight were the steps?
%   * How did we take the pulse?
%   * Was there setup? (Calm down time)
%   * Did each person do the experiment?
Five minutes before each experimental trial both of our test participants remained relatively inactive
to achieve their resting heart rate. During the experiment we had participants take their resting
heart rate with their middle and left fingers placed against an artery. We used a stopwatch and had
participants record their beats per \SI{30}{seconds} (later we would multiply this number by \SI{2}
to calculate \SI{}{beat/minute}.) \\

During our trials we had both a caffeine drinker and a non-caffeine drinker perform the experiment
in parallel on the same step, side-by-side. The step was (\SI{19}{cm} tall. The experiment was done
with a cadence of \SI{1}{step/second}. \\

At the beginning of each trial a group member, designated as the timer, started the stopwatch and
said "start" at the same time, the participants would then start stepping. At the end of each trial
the timer would say "stop" and stop the stopwatch at the same time. \\

During the part of the experiment when participants were stepping on the stair, they were encouraged
to step in unison, landing their feet on the stair and down again at the same time. For the majority
of the two participants stayed in sync. \\

After performing the experiment we traded results with four other groups to come up with our final
data set. \\

%----------------------------------------------------------------------------------------
%	SECTION 3
%----------------------------------------------------------------------------------------
\section{Results}
See Figure~\ref{fig:graph} on page~\pageref{fig:graph} and Figure~\ref{fig:table} on page~\pageref{fig:table}.

\begin{python}
print "\\begin{figure}[h]"
print "\\caption{Differences between resting heart rate and final heart rate}"

# Shell out to python to print our table
from lab_report_1 import latex_table
print latex_table()

print "\\label{fig:table}"
print "\\end{figure}"
\end{python}

\begin{sympysilent}
from lab_report_1 import latex_graph

graph = latex_graph()
\end{sympysilent}
\begin{figure}[h]
    \sympyplot[width=.7\textwidth]{graph}
    \caption{Data collected from all five groups displaying heart rate recovery time. The average
    recovery time is lower for non-caffeine drinkers compared to caffeine drinkers.}
    \label{fig:graph}
\end{figure}


%----------------------------------------------------------------------------------------
%	SECTION 4
%----------------------------------------------------------------------------------------
\section{Conclusion}
The data we collected supported our hypothesis and suggested that non-caffeine drinkers have a lower
time to recovery that caffeine drinkers. Except for Group 2, non-caffeine drinkers had a quicker or
equal time of recovery compared to caffeine drinkers. \\

The average recovery time for non-caffeine drinkers was \SI{68}{seconds} less than the
average recovery time for caffeine drinkers (See Figure~\ref{fig:graph} on page~\pageref{fig:graph}.) \\

While it does not directly support our hypothesis, it is interesting that the average difference
between resting heart rate and final heart rate was less for non-caffeine drinkers (See
figure~\ref{fig:table} on page~\pageref{fig:table}).
In this experiment we cannot say whether relative difference between beginning and ending heart rate
has any affect on recovery time, but this may be interesting subject for further research. \\

\subsection{Experimental Error}
While our data supports our hypothesis, we cannot ignore the possibility of experimental error.
Our group collected data using the methods described in section~\ref{sec:methods}, but we do not
know if the rest of the groups we collected data from used the methods. For example, it is possible
that other groups had different step cadences or that their steps were not \SI{19}{cm} tall.
Our group also used the same time measuring device for all of our trials and we cannot be certain that
other groups' stopwatches measured time in the same ways ours did.

% Statement on whether hypothesis is correct or incorrect
% Words to use:
%   * suggest
%   * support
%   * hypothesis
%
% Words to avoid:
%   * prove
%   * outlier
%   * significance

\end{document}
