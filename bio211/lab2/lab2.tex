%%%%%%%%%%%%%%%%%%%%%%%%%%%%%%%%%%%%%%%%%
% Original author:
% Linux and Unix Users Group at Virginia Tech Wiki 
% (https://vtluug.org/wiki/Example_LaTeX_chem_lab_report)
%
% License:
% CC BY-NC-SA 3.0 (http://creativecommons.org/licenses/by-nc-sa/3.0/)
%
%%%%%%%%%%%%%%%%%%%%%%%%%%%%%%%%%%%%%%%%%

%----------------------------------------------------------------------------------------
%	PACKAGES AND DOCUMENT CONFIGURATIONS
%----------------------------------------------------------------------------------------

\documentclass{article}

\usepackage{mhchem} % Package for chemical equation typesetting
\usepackage{siunitx} % Provides the \SI{}{} command for typesetting SI units
\usepackage{sympytex}
\usepackage{python}
\usepackage{astron}
\usepackage{enumitem}
\usepackage[left=3cm, top=3cm, bottom=3cm, right=3cm]{geometry}


\usepackage{graphicx} % Required for the inclusion of images

\setlength\parindent{0pt} % Removes all indentation from paragraphs

\renewcommand{\labelenumi}{\alph{enumi}.} % Make numbering in the enumerate environment by letter rather than number (e.g. section 6)

\usepackage{times} % Uncomment to use the Times New Roman font

\begin{document}

\begin{center}
    \Large{\textsc{Lab}\\[0.2cm]}
    \large{Jacques \textsc{Uber}\\}
    \today\\
\end{center}

\begin{center}
\begin{tabular}{l r}
Date Performed: & October 16, 2013 \\ % Date the experiment was performed
Partners: & Jacques Uber \\ % Partner names
& n/a \\
Instructor: & Arick Rouhe % Instructor/supervisor
\end{tabular}
\end{center}

%\begin{abstract}
% Provides a concise, clear summary of the report (summarizes
% introdution, methods, results, and discussion)
%\end{abstract}

%----------------------------------------------------------------------------------------
%	SECTION 1
%----------------------------------------------------------------------------------------
\section{Introduction}
\label{sec:introduction}
% * Provides a concise background on enzymes
A major contributer to metabolic processes are large proteins called enzymes. In cells an enzyme is
usually acting as a catalyst in an aqueous solution. The substance that an enzyme acts upon is known
as the substrate. An enzyme molecule latches onto one or more substrate molecules at a location on
the enzyme known as the active site -- this larger molecule comprised of the enzyme and the
substrate is known as the enzyme-complex \cite{lab-manual}. \\

% * Explains the major influences on the activity of enzymes
The rate at which an enzyme converts substrate to product is influenced by three major environmental
factors: the pH of the aqueous environment, the temperature of the aqueous environment, and the
ratio of enzyme to substrate. \\

The affect of pH on enzymes is a result of the shape of the protein be influenced by either excess
or lack of \ce{OH-} or \ce{H+} ions. The pH a protein is most productive at is known as the optimum
pH and every enzyme has a slightly different optimum pH. \\

The temperature of the environment will affect that rate at which an enzyme will do work. Most enzymes
have a maximum level of efficiency where temperature increases or decreases will only start to hinder
the enzymes ability to convert substrate into product. At high temperatures noncovalent forces
that hold a protein's secondary and tertiary structure in place begin to break down and the protein
begins to denature. \\

The ratio of substrate to enzyme is a factor in reaction rate as well. Most enzyme molecules react with one
substrate molecule at a time, thus when every enzyme molecule becomes busy there are no free enzyme
molecules to bind to additional substrate molecules. This saturation point is different for every
enzyme and is a function of how fast an enzyme does its work \cite{worthington}. \\


% * Explains function, activity, and influences on amylase
\newcommand{\aamylase}{$\alpha$-amylase }
\newcommand{\aamylasens}{$\alpha$-amylase}
\newcommand{\Aamylase}{$\alpha$-Amylase}
\newcommand{\onefour}{$\alpha$1,4}
\newcommand{\onesix}{$\alpha$1,6}
\newcommand{\aglucose}{$\alpha$-glucose}
\newcommand{\degreesC}[1]{$#1\,^{\circ}\mathrm{C}$}

In this lab we are considering \aamylase, an enzyme that that plays a crucial role in animals as an
enzyme that helps break down starch during digestion. \Aamylase is also used in industrial processes
as a starch liquefaction agent and as a detergent. Starch is made up of two polysaccharides:
amylose and amylopectin. Amylose is comprised of \aglucose monomers connected via a
\onefour-glycosidic bond. Amylopectin is made of similar \aglucose monomers but is more branched
than amylose due to occasional \onesix-glycosidec bonds between two glucose. \Aamylase is able to
cleave these \onefour-glycosidic bonds. \\

\Aamylase, like any enzyme, is affected by pH concentration, temperature, and substrate
concentration. \aamylase is active between pH 1.0 and roughly pH 11.5 which is important because
industrial uses of \aamylase are usually carried out at the extreme ends of the pH scale.
\cite{Nielsen}. \\

Thermostability is the ability for an enzyme to experience temperature change and still function as a
catalyst. \Aamylase is able to resist denaturing at at a broad range of temperatures. \Aamylase has
been observed breaking down starch at a wide range of temperatures (\degreesC{40} - \degreesC{100})
\cite{fitter}. \\

% * Brief overview, hypothesis, and expected outcome of experiment

The third major influence on \aamylase is the ratio of enzyme to substrate. This is the was the
focus of our hypothesis. What is the influence of \aamylase concentration on the rate of
starch digestion? We proposed that the rate at which \aamylase digests starch (amylose and
amylopectin) is proportional (linear) to the concentration of \aamylase present. \\

We will perform experimental trials that measure the time it takes for \aamylase to digest all
the starch in a solution with certain \aamylasens-to-starch concentration. \\

To make a prediction, we first define the following:
\begin{itemize}
    \item $E$: An experimental trial.
    \item $t$: The time it takes for amylase to digest all the starch in a solution.
    \item $a$: The percentage of amylase in a solution.
\end{itemize}

The equation $E(a) = t$ represents the experimental trial that used $a$ and took $t$ time to complete.

\begin{center}
    IF
        $E(a_{i}) = t_i \wedge E(a_{j}) = t_j$
    THEN
        $\frac{t_{i}}{t_{j}} = \frac{a_{i}}{a_{j}}$
\end{center}

TODO \\

%----------------------------------------------------------------------------------------
%	SECTION 2
%----------------------------------------------------------------------------------------
\section{Methods}
\label{sec:methods}
% * Described with appropriate level of detail (not a list of steps)
% * Gives enough details to allow for replication of procedure
% * Describes how iodine/starch test was used to determine rate

The experiment was performed by five separate groups. Each group prepared a set of 5 test tubes
containing a total of \SI{5}{mL} solution; the tubes contained 1:1, 1:3, 1:7, 1:15, and 1:31
concentrations of \aamylase to water, respectively. \\

A second set of corresponding tubes were setup that contained \SI{2}{mL} of \aamylase solution from
the first set of tubes and an additional 40 drops of pH 6.8 buffer solution. These tubes were then
lightly hand mixed. \\

An iodine (\ce{I_{2}KI}) plate (4 wells by 6 wells) setup. Two drops of amber colored iodine solution were placed
into each well. \\


For each test tube containing a different concentration of \aamylase the following steps were
performed:

\newlist{steps}{enumerate}{10}
\setlist[steps]{label*=\arabic*.}
\begin{steps}
    \item Mix \SI{1}{mL} of 0.5\% starch solution into the amylase dilution (This is time 0)
    \item Place two drops of the mixed solution into the first iodine well.
        \begin{enumerate}
            \item If the iodine well remains an amber yellow, stop the timer.
            \item If the iodine well remains a dark blue, wait 30 seconds and
                go to step and repeat the process using a new iodine well.
        \end{enumerate}
\end{steps}

Since we know that \ce{I_{2}KI} maintains its amber color when no starch is present and turns dark
blue when starch \emph{is} present, we can detect when the \aamylase has consumed all of the
starch in a solution. Using this method we can calculate our $t$ values for various concentrations
of \aamylase.

%----------------------------------------------------------------------------------------
%	SECTION 3
%----------------------------------------------------------------------------------------
% * Summarizes the data collected
% * Presents data in a table and in a bar graph or line graph
% * Describes important trends and data necessary for the discussion section
\section{Results}
\label{sec:results}

%----------------------------------------------------------------------------------------
%	SECTION 4
%----------------------------------------------------------------------------------------
\section{Discussion}
\label{sec:discussion}
% * Explains how the results link hypothesis/goals of the lab
% * Explains goal of the study and logical implication of results
% * Refer to other research in discussion of results

\bibliographystyle{cbe}

\bibliography{lab2}

\end{document}
